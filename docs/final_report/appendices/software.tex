\section{Software and Program Listings}
\label{sec:software}

\subsection{Code Structure}

The project consists of two main components: the backend service (server) and the WeChat Mini Program frontend (miniprogram). Below is an overview of the code structure for each part.

\paragraph{Backend: server/}
\begin{verbatim}
server/
├── app.py                  # Main backend application
├── start.py                # Entry point for server startup
├── requirements.txt        # Python dependencies
├── README.md               # Backend documentation
├── dao/                    # Data access layer (database.py, etc.)
├── service/                # Core business logic (chat, event, mood, analysis services)
│   ├── chat_langgraph_optimized.py  # Main LLM workflow engine
│   ├── event_service.py
│   ├── mood_service.py
│   └── ...
├── utils/                  # Utility functions (chat_logger, extract_json, etc.)
├── prompt/                 # Prompt templates for LLM modules
└── data/                   # Data storage (if any)
\end{verbatim}

\paragraph{Frontend: miniprogram/}
\begin{verbatim}
miniprogram/
├── app.js, app.json        # Mini Program global config and entry
├── project.config.json     # WeChat Mini Program project config
├── README.md               # Frontend documentation
├── pages/                  # Main application pages
│   ├── events/             # Event extraction and display
│   ├── medals/             # Medal (retention) system
│   ├── mood_score/         # Mood analysis and feedback
│   ├── profile/            # User profile and settings
│   └── ...
├── components/             # Reusable UI components (e.g., event-card)
├── services/               # API and business logic (api.js, event.js)
├── images/                 # Static assets and icons
└── ...
\end{verbatim}

This modular structure separates backend logic, data management, and LLM workflow (server/) from the user interface and client-side logic (miniprogram/), supporting maintainability and scalability.

