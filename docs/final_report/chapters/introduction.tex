\section{Introduction}
\label{sec:introduction}

Mental health challenges have emerged as a critical global concern, particularly exacerbated by the accelerated pace of modern life and the intensification of information overload. The prevalence of mental health issues has created an unprecedented demand for psychological support services, yet traditional mental health counseling continues to face significant barriers including exorbitant costs, limited resource availability, and substantial access obstacles. Individuals experiencing subclinical mental health problems frequently encounter difficulties in obtaining timely and professional assistance, creating a substantial gap between need and service provision.

The psychological service landscape in China exemplifies these challenges through a pronounced supply-demand imbalance. Professional mental health resources remain predominantly concentrated in tier-one cities, while the average cost per consultation Furthermore, traditional cultural values and societal norms continue to stigmatize mental health discussions, creating additional barriers that deter individuals from seeking offline professional assistance. The rapid advancement of large language models, exemplified by GPT-4 and DeepSeek, has introduced new possibilities for simulating empathetic emotional companionship, particularly within nonmedical contexts where such interventions may prove beneficial.

This project presents "Introspection: Empowering Mental Health Support with Agentic AI", a comprehensive mental health support system integrated with a psychological dialogue chatbot and psychological assessment tools.
This system leverages artificial intelligence technology and mobile applications, with the aim of providing users with personalized, accessible, and cost-effective mental health services. The system represents a novel approach to addressing the accessibility gap in mental health services through the application of agentic AI technologies.

The system architecture integrates several innovative components:
\begin{itemize}
\item An Agentic Retrieval-Augmented Generation (RAG) framework that enables context-aware responses through a sophisticated query-rewrite-retrieve pipeline
\item Mood analysis functionality with Behavioral Cognitive Therapy(BCT) design, dynamic triggering mechanisms and user feedback systems
\item Event extraction capabilities powered by LLMs for structured recording of user experiences
\item Report generation functionality which provides a summary based on history dialog, mood and event analysis through chatbot
\end{itemize}

 The system is deployed on the WeChat Mini Program platform and leverages Meta-Llama-3.1-8B-Instruct as the core large language model. The architecture of the system comprises a collaborative framework of multiple AI agents, designed to deliver holistic mental health support. These agents include an empathetic conversational agent, a mood analysis agent, and an event extraction agent, each specializing in distinct dimensions of mental health care. Together, they form a synergistic system capable of adapting to the unique needs of individual users, thereby offering highly personalized assistance.

The project addresses several critical challenges in the current mental health support landscape. First, it confronts the geographical and economic
barriers that limit access to professional psychological services. Second, it addresses the cultural stigma associated with mental health discussions by
providing an anonymous, judgment-free environment for emotional expression. Third, it leverages the scalability of AI technologies to provide consistent, 24/7 support that traditional human-based services cannot match. The significance of this research extends beyond technological innovation, contributing to the broader understanding of how AI systems can be effectively deployed in mental health support contexts. While not intended to replace professional mental health services, the system aims to provide accessible, scalable support for individuals with subclinical needs, particularly in contexts where traditional mental health resources are limited or stigmatized.

The remainder of this report is organized as follows: Section 2 provides comparative analysis of existing solutions and identifies specific challenges addressed by this research. Section 3 and 4 presents the system's architectural design and technical implementation. Section 5 describes the evaluation methodology and presents preliminary results. Section 6 discusses implications and outlines future research directions.