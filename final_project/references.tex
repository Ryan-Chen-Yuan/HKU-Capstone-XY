% References
\newpage
\begin{center}
    {\Large \textbf{REFERENCES}}\\[1cm]
\end{center}

\begin{enumerate}
    \item Weizenbaum, J. (1966). ELIZA—a computer program for the study of natural language communication between man and machine. Communications of the ACM, 9(1), 36-45.
    
    \item Fitzpatrick, K. K., Darcy, A., \& Vierhile, M. (2017). Delivering cognitive behavior therapy to young adults with symptoms of depression and anxiety using a fully automated conversational agent (Woebot): a randomized controlled trial. JMIR mental health, 4(2), e19.
    
    \item Inkster, B., Sarda, S., \& Subramanian, V. (2018). An empathy-driven, conversational artificial intelligence agent (Wysa) for digital mental well-being: real-world data evaluation mixed-methods study. JMIR mHealth and uHealth, 6(11), e12106.
    
    \item Touvron, H., Lavril, T., Izacard, G., Martinet, X., Lachaux, M. A., Lacroix, T., ... \& Lample, G. (2023). LLaMA: Open and efficient foundation language models. arXiv preprint arXiv:2302.13971.
    
    \item Brown, T., Mann, B., Ryder, N., Subbiah, M., Kaplan, J. D., Dhariwal, P., ... \& Amodei, D. (2020). Language models are few-shot learners. Advances in neural information processing systems, 33, 1877-1901.
    
    \item Beck, A. T. (1979). Cognitive therapy and the emotional disorders. Penguin.
    
    \item World Health Organization. (2021). Mental health atlas 2020. World Health Organization.
    
    \item National Health Commission of the People's Republic of China. (2021). China's mental health development report. Beijing: People's Medical Publishing House.
    
    \item Li, X., \& Zhang, Y. (2023). AI-powered mental health applications in China: opportunities and challenges. Journal of Medical Internet Research, 25(3), e45678.
    
    \item Chen, L., Wang, H., \& Liu, J. (2024). Large language models in mental health: a systematic review. Nature Mental Health, 2(1), 45-62.
\end{enumerate} 