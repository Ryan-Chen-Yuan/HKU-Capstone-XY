\section{Analysis of Problem}
\label{sec:problem_analysis}

\subsection{Market Analysis}

The Chinese mental health market faces several critical challenges:

\begin{enumerate}
    \item \textbf{Supply-Demand Imbalance}: There is a significant shortage of qualified mental health professionals in China, with the demand far exceeding available resources.
    
    \item \textbf{Geographic Concentration}: Professional mental health services are primarily concentrated in tier-one cities, making access difficult for users in smaller cities and rural areas.
    
    \item \textbf{High Cost Barrier}: Traditional counseling sessions cost between ¥500-1000 per session, creating a significant financial barrier for many potential users.
    
    \item \textbf{Cultural Stigma}: Traditional Chinese cultural values often stigmatize mental health discussions, preventing many individuals from seeking professional help.
\end{enumerate}

\subsection{Competitive Analysis}

Our research analyzed three representative AI mental health assistants:

\begin{enumerate}
    \item \textbf{Tencent Yuanbao}: A general-purpose AI assistant with strong natural language understanding but lacks specialized mental health modules.
    
    \item \textbf{Youper}: An international AI-driven mental health app that combines CBT techniques with AI chatbot functionality, but lacks Chinese localization.
    
    \item \textbf{Qingzhi Planet}: A comprehensive Chinese mental health application with multimodal emotion recognition and structured therapeutic approaches.
\end{enumerate}

\subsection{Technical Challenges}

The development of an AI-powered mental health assistant presents several technical challenges:

\begin{enumerate}
    \item \textbf{Emotional Intelligence}: Ensuring the AI can provide empathetic and appropriate responses to sensitive mental health topics.
    
    \item \textbf{Safety and Ethics}: Implementing proper safeguards for crisis situations and ensuring user privacy.
    
    \item \textbf{Personalization}: Creating tailored experiences based on individual user needs and emotional states.
    
    \item \textbf{Integration}: Seamlessly combining multiple AI modules (chatbot, assessment, community) into a cohesive user experience.
\end{enumerate} 